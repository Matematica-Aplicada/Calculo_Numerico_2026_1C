\documentclass[12pt]{article}
\usepackage{graphicx,amsmath,amsfonts,amssymb,epsfig,euscript,enumerate}
\usepackage[T1]{fontenc}
\usepackage[utf8x]{inputenc}

\newtheorem{exercise}{Ejercicio}
\newcommand{\bej}{\begin{exercise}\rm}
\newcommand{\fej}{\end{exercise}}

\newcommand{\R}{\mathbb{R}}
\newcommand{\C}{\mathbb{C}}
\def\dt{\Delta t}
\def\dx{\Delta x}

\topmargin-2cm \vsize 29.5cm \hsize 21cm
\setlength{\textwidth}{16.75cm}\setlength{\textheight}{23.5cm}
\setlength{\oddsidemargin}{0.0cm}
\setlength{\evensidemargin}{0.0cm}

\begin{document}
\centerline{{\small Universidad de Buenos Aires - Facultad de Ciencias Exactas y Naturales - Depto. de Matemática}}
 
 \vskip 0.2cm
 \hrulefill
 \vskip 0.2cm

 \centerline{{\bf\Huge {\sc Elementos de Cálculo Numérico}}}
 \vskip 0.2cm
 \centerline{\ttfamily Primer Cuatrimestre 2026}
 \hrulefill

 \bigskip
 \centerline{\bf Práctica N$^\circ$ 5: Ecuaciones Diferenciales Ordinarias}
 \bigskip

\section*{Discretización de derivadas mediante diferencias finitas}

\begin{exercise}[Análisis de error en aproximaciones de derivadas]\label{ej_difer}
Para las siguientes discretizaciones de la derivada primera, halle una expresión para el error local y señale las hipótesis de suavidad necesarias sobre la función $u$ para que el orden de precisión sea el indicado en cada caso.
\[
\begin{array}{r l c l}
 u'(x) \sim  &  \frac{u(x+h)-u(x)}{h}  &  \mbox{(diferencia forward)} & O(h) \\[0.25cm]
 u'(x) \sim  &  \frac{u(x)-u(x-h)}{h}  &  \mbox{(diferencia backward)} & O(h) \\[0.25cm]
 u'(x) \sim  &  \frac{u(x+h)-u(x-h)}{2h}  &  \mbox{(diferencia centrada)} & O(h^2)
\end{array}
\]

Repita el análisis para la discretización con diferencias centradas de la derivada segunda:
\[
u''(x) \sim \frac{u(x+h)-2u(x)+u(x-h)}{h^2}, \quad \quad \mbox{de orden} \; O(h^2).
\]
¿Qué hipótesis de suavidad es necesaria en este último caso?
\end{exercise}

\begin{exercise}[Diferencias finitas de orden alto]\label{x+2h}
Las siguientes son versiones de orden 2 de las diferencias forward o backward, que utilizan nodos a un solo lado de $x$.
\begin{enumerate}
\item Verifique que la siguiente fórmula para la derivada primera tiene orden $O(h^2)$:
\[
u'(x) \sim -\frac{1}{h}\left( \frac{3}{2}u(x)-2u(x+h)+\frac{1}{2}u(x+2h) \right).
\]
\item Halle una fórmula de aproximación para la derivada segunda $u''(x)$ que utilice los valores de $u$ en $x$, $x+h$ y $x+2h$. ¿Cuál es el orden que resulta en este caso?
\end{enumerate}
\end{exercise}

\section*{Problemas de valores iniciales}

\begin{exercise}
Dada una constante $a>0$, considere el problema de valores iniciales para $t>0$:
\[
y'(t) = -a y(t), \quad \quad  y(0)=1.
\]
Para cada paso temporal $\Delta t$ fijo se consideran las discretizaciones:
\[
\begin{array}{r l l}
      \frac{y^{n+1} - y^n}{\Delta t} = & -a y^n & \mbox{Euler explícito} \\[0.5cm]
      \frac{y^{n+1} - y^n}{\Delta t} = & -a y^{n+1} & \mbox{Euler implícito} \\[0.5cm]
      \frac{y^{n+1} - y^n}{\Delta t} = & -a \left( \frac{1}{2} y^{n+1} + \frac{1}{2} y^{n} \right) & \mbox{Adams-Moulton de 1 paso}
\end{array}
\]
Demuestre que:
\begin{enumerate}[(a)]
\item La solución exacta verifica $y(t) \to 0$ cuando $t \to \infty$.
\item Para Euler explícito se tiene $|y_n| \to 0$ si $\Delta t < 2/a$, e $|y^n| \to \infty$ si $\Delta t > 2/a$.
\item Para Euler implícito y el método de Adams-Moulton de 1 paso, $y_n \to 0$ para todo $h$.
\item Probar que todos los métodos son consistentes y por lo tanto convergentes.
\end{enumerate}
\end{exercise}

\begin{exercise}
Dada la ecuación $y' = f(t,y)$:
\begin{enumerate}[(a)]
\item Deduzca el método de Taylor de orden 2.
\item Considere ahora la familia de métodos de Runge-Kutta de 2 etapas (RK2):
\begin{align*}
k_1 &= f(t_n, y_n) \\
k_2 &= f(t_n + \alpha h, y_n + \beta h k_1) \\
y_{n+1} &= y_n + h(b_1 k_1 + b_2 k_2)
\end{align*}
Determine los coeficientes $\alpha, \beta, b_1, b_2$ para que el método coincida con el desarrollo de Taylor de orden 2.
\item Demuestre que el método resultante es consistente de orden 2.
\end{enumerate}
\end{exercise}

\begin{exercise}[Problema stiff]
Considere el PVI $y' = \lambda(y - \cos(t)) - \sin(t)$ con $y(0) = 1$ y $\lambda > 0$.
\begin{enumerate}[(a)]
\item Compruebe que la solución exacta es $y(t) = \cos(t)$.
\item Identifique la constante de Lipschitz de la función $f(t,y) = \lambda(y - \cos(t)) - \sin(t)$.
\item Utilizando la cota de error para el método de Euler explícito:
\[
|e_n| \leq \frac{M h}{2L} (e^{L(t_n - t_0)} - 1),
\]
discuta qué sucede con la fiabilidad de la cota cuando $L \to \infty$.
\item ¿Significa una $L$ grande que el método no converge? Justifique.
\end{enumerate}
\textbf{Nota}: Aquí se nota que aunque el método sea convergente ($h \to 0$), para un $h$ práctico, el error puede ser enorme si $L$ es grande.
\end{exercise}

\bej
Demuestre que al aplicar el método de Euler explícito a la ecuación $y' = \sqrt{y}$ con $y(0)=0$ el método resultante es consistente pero no es estable. 
\fej

\begin{exercise}[Condición de la raíz para estabilidad]
Dada la ecuación cuadrática $z^2 + b z +c =0$ con $b$ y $c$ en $\mathbb{R}$, demuestre que las raíces están en el círculo unitario si y solo si $|c| \leq 1$ y $|b| \leq 1+c$. \textbf{Nota}: Este resultado facilita el análisis de estabilidad de esquemas multipaso de 2 pasos.
\end{exercise}

\begin{exercise}[Método de Adams-Bashforth]
Analizar la convergencia y calcular el órden del siguiente método de Adams-Bashforth:
\[
y_{n+3}-y_{n+2}=\frac{h}{12}(23f_{n+2}-16f_{n+1}+5f_{n})
\]
\end{exercise}


\begin{exercise}[Método BDF de orden 2]
Los métodos BDF (Backward Differentiation Formulas) se construyen aproximando la derivada $y'(t_{n+1})$ mediante un polinomio interpolador que pasa por $y_{n+1}$ y pasos anteriores.
\begin{enumerate}
\item Usando el resultado del Ejercicio~\ref{x+2h} (adaptando la fórmula de diferencias finitas hacia atrás), deduzca la fórmula del método BDF de 2 pasos:
\[
\frac{3}{2} y_{n+1} - 2 y_n + \frac{1}{2} y_{n-1} = h f(t_{n+1}, y_{n+1}).
\]
\item Muestre que el método es implícito y tiene orden de consistencia 2.
\item Analice la 0-estabilidad del método hallando las raíces del polinomio característico asociado.
\item (Opcional) Determine la región de estabilidad absoluta del método.
\end{enumerate}
\end{exercise}

\begin{exercise}[Oscilador armónico discreto]
Considere el problema:
\[
y''(t) = -a y(t), \quad \quad y(0)=1, \quad y'(0)=0, \quad \quad 0<t<T_f,
\]
y la discretización explícita de 2 pasos:
\[
\frac{y^{n+1} -2y^n + y^{n-1}}{(\Delta t)^2} = -a y^n.
\]
Muestre que $|y^n| \to \infty$ si $\Delta t>2/\sqrt{a}$, y que en caso contrario $|y^n|$ permanece acotado. \textbf{Sugerencia}: Reemplace $y_n = \lambda^n$ y resuelva una ecuación de recurrencia para $\lambda$.
\end{exercise}

\section*{Problemas de valores de contorno}

\begin{exercise}\label{ejer_dirichlet}
Dado $\sigma>0$, se desea resolver numéricamente la ecuación de Reacción-Difusión (lineal) en una dimensión con condiciones de borde de tipo Dirichlet:
\begin{equation}
\label{laplaciano_dirichlet_ej}
\left\lbrace
  \begin{array}{r l c}
      -u_{xx}(x) + \sigma u(x) = & f(x), & \mbox{para } x \in (0,1) \\
      u(0) = & \alpha & \\
      u(1) = & \beta. &
      \end{array}
    \right.
\end{equation}
Para ello se considera la malla uniforme $\{ x_j = hj, \; j=0,1,2,\ldots, m+1 \}$ con $h=1/(m+1)$.
Para los puntos de la malla $x_j \in (0,1)$, el esquema de diferencias centradas para la derivada segunda (Ejercicio~\ref{ej_difer}) conduce al sistema de ecuaciones:
\[
-\frac{u_{j-1} - 2u_{j} + u_{j+1}}{h^2} + \sigma u_j = f(x_j) \quad \mbox{para} \; j=1,2,3,\ldots, m.
\]
Utilizando las condiciones de borde $u_0 = \alpha$, $u_{m+1}=\beta$ se obtiene el sistema lineal $A^h u^h = F^h$,
donde $u^h=[u_1,u_2,\ldots,u_m]^T$ es el vector de incógnitas.

\begin{enumerate}[(a)]
\item Escriba explícitamente la matriz $A^h$ y el vector $F^h$. Verifique que $A^h$ es simétrica y tridiagonal.
\item Se define el error de truncamiento local $\tau^h \in \mathbb{R}^m$ como el residuo al insertar la solución exacta $u(x)$ en el esquema numérico:
\[
\tau^h_j =  \left( -\frac{u(x_{j-1}) - 2u(x_j) + u(x_{j+1})}{h^2} + \sigma u(x_j) \right) - f(x_j).
\]
Demuestre que si $u \in C^4[0,1]$, entonces existe $C > 0$ tal que $\|\tau^h\|_\infty \leq C h^2$.

\item Demuestre que la matriz $A^h$ es estrictamente diagonal dominante (por filas) si $\sigma > 0$.
\item Utilice el teorema de los círculos de Gershgorin para mostrar que los autovalores de $A^h$ son positivos y están acotados inferiormente por $\sigma$. Concluya que $\| (A^h)^{-1} \|_\infty \leq 1/\sigma$.
\item Combinando los resultados anteriores, demuestre la convergencia del método, es decir, que el error global $E^h_j = u(x_j) - u^h_j$ satisface $\| E^h \|_\infty \to 0$ cuando $h \to 0$.
\end{enumerate}
\end{exercise}

\begin{exercise}[Condiciones de Neumann]
Se desea resolver numéricamente la ecuación de Poisson en una dimensión con condiciones de Neumann en $x=0$ y de Dirichlet en $x=1$:
\begin{equation}
\label{laplaciano_neumann_ej}
\left\lbrace
  \begin{array}{r l r}
      u_{xx}(x) = & f(x), & \mbox{para } x \in (0,1) \\
      u_x(0) = & 0 & \\
      u(1) = & 0. &
      \end{array}
    \right.
\end{equation}
Obtenga matrices para el problema discreto, si para la condición de Neumann $u_x(0)=0$ se realizan las siguientes aproximaciones:
\begin{enumerate}
\item $u_1 - u_0 = 0$ (diferencias forward)
\item $\frac{3}{2}u_0 - 2u_1 + \frac{1}{2}u_2 = 0$ (diferencias forward de orden 2 del Ejercicio~\ref{x+2h})
\end{enumerate}

¿Cuál es el orden del error de truncamiento en el interior y en el punto $x=0$ en cada caso?
\end{exercise}

\begin{exercise}[Estabilidad por análisis de Fourier]\label{ej_estabilidad_norma2}
Dado $\sigma > 0$, considere la ecuación de reacción-difusión con condiciones periódicas:
\[
\left\lbrace
  \begin{array}{r l c}
      -u''(x) + \sigma u(x) = & f(x), & \mbox{para } x \in (0,1), \\
      u(0) = u(1), & \quad u'(0) = u'(1). &
  \end{array}
\right.
\]
Se toma la malla uniforme $x_j = jh$, $j=0,1,\ldots,N-1$, con $h=1/N$.
\begin{enumerate}[(a)]
\item Aplicando diferencias centradas para $u''(x_j)$, muestre que el esquema numérico equivale a
\[
\frac{-u_{j-1} + 2\, u_j - u_{j+1}}{h^2} + \sigma\, u_j = f(x_j), \quad j=0,1,\ldots,N-1,
\]
con índices tomados módulo $N$ (es decir, $u_{-1} = u_{N-1}$ y $u_N = u_0$).

\item Para hallar los autovalores de la matriz $A_N$ del sistema resultante $A_N u^h = F^h$, proponga el modo de Fourier discreto $w^{(k)}_j = e^{2\pi i kj/N}$ como autovector. Sustituyendo en la ecuación homogénea y usando la identidad $e^{i\theta}+e^{-i\theta} = 2\cos\theta$, demuestre que $w^{(k)}$ es autovector con autovalor
\[
\lambda_k = \frac{4}{h^2}\sin^2\!\left(\frac{\pi k}{N}\right) + \sigma, \quad k=0,1,\ldots,N-1.
\]

\item Dado que $A_N$ es simétrica real, $\|A_N^{-1}\|_2 = 1/\lambda_{\min}$. Muestre que $\lambda_{\min} = \sigma$ y concluya que existe $C > 0$ independiente de $h$ tal que $\|A_N^{-1}\|_2 \leq C$, lo que prueba la estabilidad.
\end{enumerate}
\end{exercise}

\begin{exercise}[Ecuación de advección-reacción-difusión periódica]
Dados $b \in \R$ y $\sigma > 0$, considere la ecuación de advección-reacción-difusión con condiciones periódicas:
\[
\left\lbrace
  \begin{array}{r l c}
      -u''(x) + b\,u'(x) + \sigma\, u(x) = & f(x), & \mbox{para } x \in (0,1), \\
      u(0) = u(1), & \quad u'(0) = u'(1). &
  \end{array}
\right.
\]
Se toma la malla uniforme $x_j = jh$, $j=0,1,\ldots,N-1$, con $h=1/N$.
\begin{enumerate}[(a)]
\item Utilizando diferencias centradas para $u''(x_j)$ y $u'(x_j)$ (Ejercicio~\ref{ej_difer}), muestre que el esquema numérico se escribe como $A_N u^h = F^h$, donde la ecuación para el nodo $j$-ésimo es
\[
\frac{-u_{j-1} + 2u_j - u_{j+1}}{h^2} + b\,\frac{u_{j+1}-u_{j-1}}{2h} + \sigma\, u_j = f(x_j), \quad j = 0,\ldots,N-1,
\]
con índices módulo $N$.

\item Aplique el método de Fourier del Ejercicio~\ref{ej_estabilidad_norma2}. Usando además la identidad $e^{i\theta}-e^{-i\theta} = 2i\sin\theta$, demuestre que $w^{(k)}$ es autovector de $A_N$ con autovalor
\[
\lambda_k = \frac{4}{h^2}\sin^2\!\left(\frac{\pi k}{N}\right) + \frac{ib}{h}\sin\!\left(\frac{2\pi k}{N}\right) + \sigma, \quad k=0,1,\ldots,N-1.
\]
Observe que si $b \neq 0$, los autovalores son complejos (la matriz $A_N$ no es simétrica).

\item Muestre que $|\lambda_k| \geq \operatorname{Re}(\lambda_k) \geq \sigma > 0$ para todo $k$. Muestre que la matriz $A_N$ es normal y que por lo tanto de modo que $\|A_N^{-1}\|_2 = 1/\min_k |\lambda_k|$. Concluya que existe $C > 0$ independiente de $h$ tal que $\|A_N^{-1}\|_2 \leq C$.
\end{enumerate}
\end{exercise}

\begin{exercise}[Ecuación de Helmholtz periódica]
Dado $\omega > 0$, considere la ecuación de Helmholtz con condiciones periódicas:
\[
\left\lbrace
  \begin{array}{r l c}
      -u''(x) - \omega^2 u(x) = & f(x), & \mbox{para } x \in (0,1), \\
      u(0) = u(1), & \quad u'(0) = u'(1). &
  \end{array}
\right.
\]
Aplique el método de Fourier del Ejercicio~\ref{ej_estabilidad_norma2} a la discretización por diferencias centradas para demostrar que los autovalores son
\[
\lambda_k = \frac{4}{h^2}\sin^2\!\left(\frac{\pi k}{N}\right) - \omega^2, \quad k=0,1,\ldots,N-1.
\]
\begin{enumerate}[(a)]
\item Muestre que la matriz no es definida positiva. 
\item Muestre que la matriz es singular si y solo si $\omega^2$ coincide con algún autovalor del laplaciano discreto periódico $\frac{4}{h^2}\sin^2\!\left(\frac{\pi k}{N}\right)$.
\item Demuestre que si $\omega \neq 2n\pi$ para todo $n \in \mathbb{N}$ (es decir, $\omega$ no es autovalor de $-u''$ con condiciones periódicas en $[0,1]$), entonces existe una constante $C > 0$ independiente de $h$ tal que $\|A_N^{-1}\|_2 \leq C$.
\end{enumerate}
\end{exercise}

\begin{exercise}[Estabilidad en norma 2: caso Dirichlet]\label{ej_estabilidad_dirichlet}
Considere la discretización por diferencias centradas de $-u''(x) = f(x)$ en $(0,1)$ con condiciones de Dirichlet $u(0) = \alpha$, $u(1) = \beta$, sobre la malla $x_j = jh$, $h = 1/(m+1)$. El sistema resultante es $A^h u^h = F^h$ con $A^h \in \R^{m \times m}$.

El problema de autovalores $A^h v = \lambda v$ conduce, para $j=1,\ldots,m$, a la ecuación
\[
-v_{j-1} + 2\, v_j - v_{j+1} = \lambda h^2\, v_j,
\]
con condiciones de borde $v_0 = 0$ y $v_{m+1} = 0$.
\begin{enumerate}[(a)]
\item Proponga el ansatz $v_j = r^j$ y obtenga la ecuación característica $r^2 - \mu\, r + 1 = 0$ con $\mu = 2 - \lambda h^2$. Deduzca que la solución general es $v_j = c_1 e^{ij\theta} + c_2 e^{-ij\theta}$ con $\cos\theta = \mu/2$.
\item Usando $v_0 = 0$, muestre que $v_j = C \sin(j\theta)$. Imponga $v_{m+1}=0$ para obtener $\theta_k = \dfrac{k\pi}{m+1}$, $k=1,\ldots,m$, y concluya que los autovalores son
\[
\lambda_k = \frac{4}{h^2}\sin^2\!\left(\frac{k\pi h}{2}\right), \quad k=1,\ldots,m.
\]
\item Muestre que $\lambda_{\min} = \lambda_1 \to \pi^2$ cuando $h \to 0$. Concluya que existe $C > 0$ independiente de $h$ tal que $\|(A^h)^{-1}\|_2 \leq C$, lo que prueba la estabilidad en norma~2.
\end{enumerate}
\end{exercise}

\begin{exercise}[Partícula cuántica en un anillo: autovalores discretos vs.\ continuos]
La ecuación de Schrödinger estacionaria para una partícula libre en un anillo de longitud $L=1$ (con $\hbar = 2m = 1$) es el problema de autovalores con condiciones periódicas:
\[
\left\lbrace
  \begin{array}{r l c}
      -u''(x) = & \lambda\, u(x), & \mbox{para } x \in (0,1), \\
      u(0) = u(1), & \quad u'(0) = u'(1), &
  \end{array}
\right.
\]
donde los autovalores $\lambda$ representan los niveles de energía del sistema.
\begin{enumerate}[(a)]
\item Resuelva el problema continuo y muestre que los autovalores son $\lambda_k = (2\pi k)^2$, $k = 0, 1, 2, \ldots$, con autofunciones $u(x) = e^{\pm 2\pi i k x}$ (multiplicidad~2 para $k \geq 1$).

\item Usando los resultados del Ejercicio~\ref{ej_estabilidad_norma2} y la expansión en Taylor de $\sin(\xi)$, demuestre que para cada $k$ fijo,
\[
\lambda_k^h = (2\pi k)^2 - \frac{(2\pi k)^4}{12}\, h^2 + O(h^4) \quad \mbox{cuando } h \to 0.
\]
Es decir, la aproximación de cada autovalor individual es de orden~2.

\item ¿Qué sucede con la calidad de la aproximación para $k$ grande (modos de alta frecuencia)? Muestre que $\lambda_k^h \leq 4/h^2$ para todo $k$, mientras que $\lambda_k$ crece sin cota. Concluya que la discretización solo puede aproximar bien los primeros $O(N)$ autovalores.
\end{enumerate}
\end{exercise}


\end{document}






