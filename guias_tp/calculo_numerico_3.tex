\documentclass[12pt]{article}
\usepackage{graphicx,amsmath,amsfonts,amssymb,epsfig,euscript,enumerate}
\usepackage[T1]{fontenc}
\usepackage[utf8x]{inputenc}

\newtheorem{exercise}{Ejercicio}
\newcommand{\bej}{\begin{exercise}\rm}
\newcommand{\fej}{\end{exercise}}

\newcommand{\R}{\mathbb{R}}
\newcommand{\C}{\mathbb{C}}
\def\dt{\Delta t}
\def\dx{\Delta x}

\topmargin-2cm \vsize 29.5cm \hsize 21cm
\setlength{\textwidth}{16.75cm}\setlength{\textheight}{23.5cm}
\setlength{\oddsidemargin}{0.0cm}
\setlength{\evensidemargin}{0.0cm}

\begin{document}
\centerline{{\small Universidad de Buenos Aires - Facultad de Ciencias Exactas y Naturales - Depto. de Matemática}}
 
 \vskip 0.2cm
 \hrulefill
 \vskip 0.2cm

 \centerline{{\bf\Huge {\sc Elementos de Cálculo Numérico}}}
 \vskip 0.2cm
 \centerline{\ttfamily Primer Cuatrimestre 2026}
 \hrulefill

 \bigskip
 \centerline{\bf Práctica N$^\circ$ 3: Interpolación}
 \bigskip

\noindent \textbf{Problema de Interpolación:} Dados $n+1$ puntos distintos $x_0, \ldots, x_n$ (llamados nodos) y $n+1$ valores $y_0, \ldots, y_n$, buscamos un polinomio $p(x)$ de grado menor o igual a $n$ tal que $p(x_i) = y_i$ para todo $i = 0, \ldots, n$.

\begin{exercise}[Interpolación como problema lineal]
El problema de interpolación se puede ver como una transformación lineal entre espacios vectoriales.
\begin{enumerate}[(a)]
\item Sea $\mathcal{P}_n$ el espacio de polinomios de grado a lo sumo $n$. Verifique que $\mathcal{P}_n$ es un espacio vectorial y que $\{1, x, x^2, \ldots, x^n\}$ es una base.
\item Defina el operador de evaluación $\Phi: \mathcal{P}_n \to \mathbb{R}^{n+1}$ dado por:
\[
\Phi(p) = (p(x_0), p(x_1), \ldots, p(x_n)).
\]
Demuestre que $\Phi$ es una transformación lineal.
\item Interprete el problema de interpolación como la búsqueda de la transformación inversa $\Phi^{-1}$: ¿qué representan la entrada y la salida de $\Phi^{-1}$?
\item Demuestre que si los nodos $x_0, \ldots, x_n$ son distintos, el problema de interpolación tiene solución única.

\textbf{Sugerencia}: Pruebe que $\Phi$ es inyectiva. Use que un polinomio no nulo de grado $\leq n$ tiene a lo sumo $n$ raíces.
\end{enumerate}
\end{exercise}

\begin{exercise}[Matriz de Vandermonde]
Dados nodos $x_0, x_1, \ldots, x_n$ distintos y valores $y_0, y_1, \ldots, y_n$, queremos encontrar el polinomio $p(x) = \sum_{k=0}^n a_k x^k$ que satisface $p(x_i) = y_i$.
\begin{enumerate}[(a)]
\item Escriba el sistema lineal correspondiente en forma matricial $Va = y$. La matriz $V$ se conoce como matriz de Vandermonde. 
\item Analice el condicionamiento en norma 2 utilizando la caracterización variacional del valor singular mínimo: $\sigma_{\min}(V) = \min_{a \neq 0} \frac{\|Va\|_2}{\|a\|_2}$.
   \begin{enumerate}[i.]
   \item Observe que $\|Va\|_2^2 = \sum_{i=0}^n (p(x_i))^2$, donde $p(x)$ es el polinomio con coeficientes $a$.
   Si los nodos $x_i$ cubren densamente el intervalo $[-1, 1]$, interprete esta suma como una aproximación de Riemann escalada para justificar la relación aproximada (donde $C_n = O(n)$):
   \[
   \|Va\|_2^2 \approx C_n \int_{-1}^1 (p(x))^2 dx.
   \]
   \item Considere el polinomio de prueba $p(x) = (1-x^2)^n$. Identifique sus coeficientes no nulos y demuestre usando la identidad $\sum_{k=0}^n \binom{n}{k}^2 = \binom{2n}{n}$ que la norma de los coeficientes satisface $\|a\|_2^2 \geq 4^n / (2n+1)$.
   \item Justifique que la integral $\int_{-1}^1 (1-x^2)^{2n} dx$ es pequeña (tiende a 0 con $n$) mientras que $\|a\|_2^2$ crece exponencialmente.
   \item Concluya que $\sigma_{\min}(V)$ decae exponencialmente con $n$, lo que implica que $\kappa_2(V)$ crece exponencialmente.
   \end{enumerate}
\end{enumerate}
\end{exercise}

\begin{exercise}[Polinomios base de Lagrange]
\begin{enumerate}[(a)]
\item Para los nodos $x_0 = -1, x_1 = 0, x_2 = 1$, calcule explícitamente los tres polinomios base de Lagrange:
\[
\ell_k(x) = \prod_{\substack{j=0 \\ j \neq k}}^2 \frac{x - x_j}{x_k - x_j}, \quad k = 0, 1, 2.
\]
\item Verifique que $\ell_k(x_j) = \delta_{kj}$ (propiedad de Kronecker).
\item Verifique que $\ell_0(x) + \ell_1(x) + \ell_2(x) = 1$ para todo $x$ (propiedad de partición de la unidad).
\item Use la fórmula de Lagrange para encontrar el polinomio que interpola los puntos $(-1, 2), (0, -1), (1, 4)$.
\item Evalúe $p(0.5)$ y compare con evaluación directa en los polinomios base.
\end{enumerate}
\end{exercise}

\begin{exercise}[Forma baricéntrica de Lagrange]
La forma baricéntrica es una reformulación de la interpolación de Lagrange.
\begin{enumerate}[(a)]
\item Defina los pesos baricéntricos:
\[
w_k = \frac{1}{\prod_{\substack{j=0 \\ j \neq k}}^n (x_k - x_j)}, \quad k = 0, \ldots, n.
\]
\item Demuestre que el polinomio interpolante se puede escribir como:
\[
p(x) = \frac{\sum_{k=0}^n \frac{w_k}{x - x_k} y_k}{\sum_{k=0}^n \frac{w_k}{x - x_k}}.
\]
\item Compare el costo computacional de evaluar:
   \begin{itemize}
   \item La fórmula de Lagrange estándar: $O(n^2)$ por evaluación.
   \item La forma baricéntrica: $O(n^2)$ preproceso + $O(n)$ por evaluación.
   \end{itemize}
\end{enumerate}
\end{exercise}

\begin{exercise}[Error de interpolación]
Sea $f \in C^{n+1}([a,b])$ y $p_n$ el polinomio que interpola $f$ en $n+1$ nodos $x_0, \ldots, x_n \in [a,b]$.
\begin{enumerate}[(a)]
\item Para $f(x) = e^x$ en $[0, 1]$ con nodos $x_0 = 0, x_1 = 0.5, x_2 = 1$:
   \begin{itemize}
   \item Encuentre el polinomio interpolante $p_2(x)$.
   \item Estime el error máximo en $[0, 1]$ usando la fórmula del error.
   \item Compare con el error real $|f(0.25) - p_2(0.25)|$.
   \end{itemize}
\item ¿En qué puntos $x$ el error es exactamente cero?
\end{enumerate}
\end{exercise}

\begin{exercise}[Interpolación a trozos]
Dados nodos $x_0 < \cdots < x_n$ y valores $y_i$, la interpolación lineal a trozos $s(x)$ conecta los puntos $(x_i, y_i)$ con segmentos de recta.
\begin{enumerate}[(a)]
\item Escriba la expresión de $s(x)$ en $[x_i, x_{i+1}]$ y verifique su continuidad en los nodos. Verifique que no es diferenciable en general.
\item Demuestre la cota de error $\|f - s\|_\infty \leq \frac{h^2}{8} \|f''\|_\infty$, donde $h = \max_i \Delta x_i$.
\end{enumerate}
\end{exercise}

\begin{exercise}[Splines cúbicos]
Un \textbf{spline cúbico} $s(x)$ es una función $C^2$ que coincide con un polinomio cúbico en cada subintervalo $[x_i, x_{i+1}]$ e interpola los datos.
\begin{enumerate}[(a)]
\item Cuente los grados de libertad ($4n$ coeficientes) y restricciones ($2n$ de interpolación/continuidad en extremos de intervalos, $2(n-1)$ de suavidad $C^1, C^2$). Concluya que faltan 2 condiciones de frontera (ej. ``natural'' $s''=0$, o ``sujeto'' $s'=f'$ en extremos).
\item Demuestre la propiedad variacional: El spline cúbico natural minimiza la energía de curvatura $\int_a^b [u''(x)]^2 dx$ entre todas las funciones $C^2$ que interpolan los datos.
\end{enumerate}
\end{exercise}

\begin{exercise}[Cuadratura interpolatoria] Un método muy común y general de integración numérica es interpolar el integrando, e integrar el polinomio resultante de manera analítica.
\begin{enumerate}[(a)]
\item \textbf{Regla del trapecio}: Interpole $f(x)$ en $[a, b]$ con una línea recta pasando por $(a, f(a))$ y $(b, f(b))$. Demuestre que obtiene la aproximación:
\[
\int_a^b f(x) dx \approx \frac{b-a}{2}(f(a) + f(b)).
\]
\item \textbf{Regla de Simpson}: Interpole $f(x)$ en $[a, b]$ con una parábola pasando por $x_0 = a$, $x_1 = (a+b)/2$, $x_2 = b$. Demuestre que obtiene la aproximación:
\[
\int_a^b f(x) dx \approx \frac{b-a}{6}\left(f(a) + 4f\left(\frac{a+b}{2}\right) + f(b)\right).
\]
\item Use el teorema del error de interpolación para estimar el error de cada fórmula.
\end{enumerate}
\end{exercise}

\begin{exercise}[Límite de precisión de las reglas de cuadratura]
Sea $Q_n(f) = \sum_{i=0}^n w_i f(x_i)$ una regla de cuadratura interpolatoria basada en $n+1$ nodos distintos $x_0, \ldots, x_n$.
\begin{enumerate}[(a)]
\item Si es posible elegir tanto los nodos como los pesos, ¿Cuántos grados de libertad existen? 
\item Demuestre que es imposible construir una regla de cuadratura con $n+1$ nodos que sea exacta para todos los polinomios de grado $2n+2$.\textbf{Sugerencia}: Considere el polinomio $P(x) = \prod_{i=0}^n (x-x_i)^2$.
\end{enumerate}


\end{exercise}

\begin{exercise}[Diferenciación numérica]
Muchas fórmulas de diferenciación numérica se pueden obtener derivando polinomios que interpolan los datos.
\begin{enumerate}[(a)]
\item Interpole $f(x)$ en $x_0, x_0 + h$ con el polinomio $p_1(x)$ de Lagrange.
\item Derive $p_1(x)$ para obtener la aproximación de diferencias finitas:
\[
f'(x_0) \approx \frac{f(x_0 + h) - f(x_0)}{h}.
\]
\item Interpole $f(x)$ en $x_0 - h, x_0, x_0 + h$ con el polinomio $p_2(x)$.
\item Derive $p_2(x)$ y evalúe en $x_0$ para obtener la fórmula centrada:
\[
f'(x_0) \approx \frac{f(x_0 + h) - f(x_0 - h)}{2h}.
\]
\item Use el teorema del error de interpolación para estimar el error de truncamiento de cada fórmula.
\end{enumerate}
\end{exercise} 


\subsection*{Transformada Discreta de Fourier}

Dado $x \in \mathbb{C}^N$, definimos su DFT como $\hat{x} \in \mathbb{C}^N$ con
\[
\hat{x}_k = \sum_{j=0}^{N-1} x_j\, \omega_N^{jk}, \quad k = 0, \ldots, N-1, \qquad \text{donde } \omega_N = e^{2\pi i/N}.
\]

\begin{exercise}[Aliasing]
Sea $x_j = \omega_N^{mj}$ el modo de Fourier de frecuencia $m$ muestreado en la grilla $j = 0, \ldots, N-1$.
\begin{enumerate}[(a)]
\item Demuestre que $\omega_N^{(m+N)j} = \omega_N^{mj}$ para todo $j$. Concluya que los modos de frecuencia $m$ y $m + N$ producen exactamente la misma señal discreta.
\item Más generalmente, demuestre que los modos de frecuencia $m$ y $m + kN$ son indistinguibles para todo $k \in \mathbb{Z}$.
\item Concluya que con $N$ muestras solo se pueden representar $N$ frecuencias distintas. ¿Cuáles son las frecuencias ``esencialmente distintas''?
\end{enumerate}
\end{exercise}

\begin{exercise}[Reindexación de frecuencias]
La DFT indexa las frecuencias como $k = 0, 1, \ldots, N-1$. Pero usando aliasing, las frecuencias altas se pueden reinterpretar como frecuencias negativas.
\begin{enumerate}[(a)]
\item Para $N$ par, demuestre que el modo $k = N - \ell$ (con $1 \leq \ell \leq N/2-1$) es idéntico al modo de frecuencia $-\ell$. Concluya que la DFT con índices $k = 0, \ldots, N-1$ es equivalente a usar frecuencias $k = -N/2, \ldots, N/2 - 1$.
\item Si $x \in \mathbb{R}^N$ (señal real), demuestre que $\hat{x}_{N-k} = \overline{\hat{x}_k}$. Concluya que los coeficientes de Fourier para frecuencias negativas son los conjugados de los positivos, y que por lo tanto la DFT de una señal real queda determinada por los coeficientes $\hat{x}_0, \hat{x}_1, \ldots, \hat{x}_{N/2}$.
\end{enumerate}
\end{exercise}

\begin{exercise}[Difracción por rendijas múltiples]
Una red de difracción tiene $N$ rendijas equiespaciadas con separación $d$. La amplitud compleja en el ángulo $\theta$ es proporcional a la suma $A(\theta) = \sum_{n=0}^{N-1} a_n\, e^{i n q d}$, donde $q = \frac{2\pi \sin\theta}{\lambda}$ y $a_n \in \{0, 1\}$ indica si la rendija $n$-ésima está abierta.
\begin{enumerate}[(a)]
\item Identifique $A(\theta)$ como la evaluación del polinomio $P_a(z) = \sum_{n=0}^{N-1} a_n z^n$ en $z = e^{iqd}$. Concluya que calcular la intensidad $I(\theta) = |A(\theta)|^2$ en los $N$ ángulos $q_k = \frac{2\pi k}{Nd}$ equivale a calcular la DFT de $a$.
\item Explique por qué usar la FFT para obtener el patrón de difracción completo es más eficiente que evaluar $A(\theta)$ en cada ángulo por separado.
\end{enumerate}
\end{exercise}


\begin{exercise}[La DFT como producto matricial]
\begin{enumerate}[(a)]
\item Escriba la DFT $\hat{x}_k = \sum_{j=0}^{N-1} x_j\, \omega_N^{jk}$ como un producto matricial $\hat{x} = F\, x$, donde $F \in \mathbb{C}^{N \times N}$ es la \textit{matriz de Fourier} con entradas $F_{kj} = \omega_N^{jk}$.
\item Observe que $F$ es la matriz de Vandermonde evaluada en los nodos $z_0 = 1,\, z_1 = \omega_N,\, z_2 = \omega_N^2,\, \ldots,\, z_{N-1} = \omega_N^{N-1}$:
\[
F = \begin{pmatrix}
1 & 1 & 1 & \cdots & 1 \\
1 & \omega_N & \omega_N^2 & \cdots & \omega_N^{N-1} \\
1 & \omega_N^2 & \omega_N^4 & \cdots & \omega_N^{2(N-1)} \\
\vdots & \vdots & \vdots & \ddots & \vdots \\
1 & \omega_N^{N-1} & \omega_N^{2(N-1)} & \cdots & \omega_N^{(N-1)^2}
\end{pmatrix}.
\]
\item Interprete: calcular la DFT es evaluar el polinomio $P_x(z) = \sum_{j=0}^{N-1} x_j z^j$ en las $N$ raíces de la unidad. Calcular la DFT inversa es interpolar a partir de esas evaluaciones. ¿Por qué la inversión es posible?
\end{enumerate}
\end{exercise}

\begin{exercise}[Unitariedad de la matriz de Fourier]
Sea $F$ la matriz de Fourier del ejercicio anterior.
\begin{enumerate}[(a)]
\item Demuestre la identidad de ortogonalidad: $\sum_{j=0}^{N-1} \omega_N^{jk} \overline{\omega_N^{j\ell}} = N\,\delta_{k\ell}$.

\textbf{Sugerencia}: La suma es una serie geométrica.
\item Deduzca que $F\overline{F} = NI$, y por lo tanto $F^{-1} = \frac{1}{N}\overline{F}$.
\item Escriba explícitamente la fórmula de la DFT inversa:
\[
x_j = \frac{1}{N} \sum_{k=0}^{N-1} \hat{x}_k\, \omega_N^{-jk}.
\]
Compare con la fórmula de la DFT directa. ¿En qué se diferencian?
\item Concluya que $\frac{1}{\sqrt{N}}F$ es una matriz unitaria. 
\end{enumerate}
\end{exercise}

\begin{exercise}[Teorema de Parseval discreto]
Usando que $\frac{1}{\sqrt{N}}F$ es unitaria, demuestre la identidad de Parseval:
\[
\sum_{k=0}^{N-1} |\hat{x}_k|^2 = N \sum_{j=0}^{N-1} |x_j|^2.
\]
Interprete: la energía en el dominio del tiempo es proporcional a la energía en frecuencia.
\end{exercise}

\begin{exercise}[Convolución Discreta]
Una de las propiedades más importantes de la DFT es su relación con la convolución.
\begin{enumerate}[(a)]
\item Dadas dos secuencias $u, v \in \mathbb{C}^N$, definimos su \textbf{convolución circular} $w = u * v$ como:
\[
w_k = \sum_{j=0}^{N-1} u_j v_{(k-j) \pmod N}, \quad k=0, \ldots, N-1.
\]
\item Demuestre el Teorema de la Convolución: La DFT de la convolución es el producto punto a punto de las DFTs (salvo por un factor de escala dependiente de la normalización):
\[
\widehat{(u * v)}_k = \sqrt{N} \hat{u}_k \hat{v}_k.
\]
\item Explique cómo esto permite calcular la convolución de dos vectores muy largos de manera eficiente ($O(N \log N)$) usando la Transformada Rápida de Fourier (FFT), en contraste con el costo $O(N^2)$ de la definición directa.
\end{enumerate}
\end{exercise}

\begin{exercise}[Superposición e invariancia traslacional]
Considere $N$ fuentes equiespaciadas en un intervalo $[0, L)$ con condiciones de contorno periódicas, ubicadas en $x_j = jh$ con $h = L/N$ ($j = 0, \ldots, N-1$). La fuente en $x_j$ tiene amplitud $\rho_j \in \mathbb{C}$. Cada fuente unitaria genera un potencial dado por una función $g$ que depende solo de la distancia: el potencial en $x$ debido a una fuente unitaria en $x'$ es $g(x - x')$.
\begin{enumerate}[(a)]
\item (\textbf{Invariancia traslacional}) Defina $f_m = g(mh)$ para $m = 0, \ldots, N-1$. Demuestre que el potencial en $x_n$ debido a una fuente unitaria en $x_j$ es $f_{(n-j) \bmod N}$, es decir, depende solo de $n - j$.
\item (\textbf{Linealidad}) Usando que el potencial total es la suma de las contribuciones individuales ponderadas por las amplitudes, demuestre que el potencial total en $x_n$ es $\phi_n = \sum_{j=0}^{N-1} \rho_j\, f_{(n-j) \bmod N}$. Concluya que $\phi = \rho * f$ (convolución circular).
\item Use el Teorema de la Convolución para concluir que $\phi$ se puede calcular en $O(N \log N)$ operaciones en vez de $O(N^2)$.
\end{enumerate}
\end{exercise}

\begin{exercise}[Multiplicación rápida de polinomios]
Sean $P(x) = \sum_{j=0}^{n-1} a_j x^j$ y $Q(x) = \sum_{k=0}^{n-1} b_k x^k$ dos polinomios de grado menor que $n$. Queremos calcular su producto $R(x) = P(x)Q(x)$.
\begin{enumerate}[(a)]
\item Muestre que el coeficiente $c_k$ del término $x^k$ en $R(x)$ está dado por la convolución de los coeficientes de $P$ y $Q$:
\[
c_k = \sum_{j=0}^k a_j b_{k-j}.
\]
\item Observe que el grado de $R(x)$ puede ser hasta $2n-2$. Para usar el Teorema de la Convolución Cíclica (que opera en vectores de longitud fija $N$), necesitamos ``rellenar con ceros'' (zero-padding).
Defina vectores extendidos $A, B \in \mathbb{C}^N$ con $N \ge 2n-1$ completando con ceros.
\item Describa el algoritmo completo para multiplicar polinomios usando FFT:
   \begin{enumerate}[i.]
   \item Extender coeficientes a tamaño $N$.
   \item Calcular $\text{FFT}(A)$ y $\text{FFT}(B)$.
   \item Multiplicar punto a punto.
   \item Calcular $\text{IFFT}$ del resultado.
   \end{enumerate}
\item Compare la complejidad asintótica de este método con la multiplicación clásica "todos con todos" ($O(n^2)$). ¿A partir de qué grado $n$ aproximado cree que vale la pena usar FFT?
\end{enumerate}
\end{exercise}

\subsection*{Interpolación en cuerpos finitos (*)}

\begin{exercise}[Interpolación en cuerpos finitos: definiciones básicas]
La interpolación polinomial también funciona sobre cuerpos finitos $\mathbb{Z}_p$ donde $p$ es primo.
\begin{enumerate}[(a)]
\item Verifique que el espacio de polinomios $\mathcal{P}_n(\mathbb{Z}_p) = \{a_0 + a_1 x + \cdots + a_n x^n : a_i \in \mathbb{Z}_p\}$ es un espacio vectorial sobre $\mathbb{Z}_p$. ¿Cuál es su dimensión?
\item Demuestre que un polinomio no nulo $q \in \mathcal{P}_n(\mathbb{Z}_p)$ tiene a lo sumo $n$ raíces en $\mathbb{Z}_p$.

\textit{Sugerencia}: Use inducción en $n$. Si $q(\beta) = 0$, divida $q(x)$ por el polinomio mónico $(x - \beta)$ usando el algoritmo de división para obtener $q(x) = (x - \beta)\,\tilde{q}(x)$ con $\tilde{q} \in \mathcal{P}_{n-1}(\mathbb{Z}_p)$. 
\item Dados $n+1$ puntos distintos $(x_0, y_0), \ldots, (x_n, y_n)$ con $x_i, y_i \in \mathbb{Z}_p$, demuestre que existe un único polinomio $p \in \mathcal{P}_n(\mathbb{Z}_p)$ tal que $p(x_i) = y_i$ para todo $i$.
\end{enumerate}
\end{exercise}

\begin{exercise}[Interpolación en $\mathbb{Z}_7$]
Considere los puntos $(1, 3), (2, 5), (4, 2)$ en $\mathbb{Z}_7$.
\begin{enumerate}[(a)]
\item Halle los coeficientes del polinomio interpolante $p(x) = a_0 + a_1 x + a_2 x^2$ resolviendo el sistema de Vandermonde $Va = y$ en $\mathbb{Z}_7$.
\item Evalúe $p(3)$ usando la fórmula de Lagrange (sin calcular los coeficientes).
\end{enumerate}
\end{exercise}

\begin{exercise}[Esquema de Shamir] Un esquema de compartición de secretos $(k, n)$ permite dividir un secreto entre $n$ personas de modo que cualquier $k$ de ellas puedan reconstruirlo, pero $k-1$ no pueden obtener información alguna.

El secreto es un número $s \in \mathbb{Z}_p$ (con $p$ primo grande). Se elige un polinomio $f(x) = s + a_1 x + \cdots + a_{k-1} x^{k-1} \pmod{p}$ de grado $k-1$ con $f(0) = s$. Los coeficientes $a_i$ se eligen al azar en $\mathbb{Z}_p$ siguiendo una distribución uniforme. Luego, se distribuyen las $n$ "porciones" (shares): $(1, f(1)), (2, f(2)), \ldots, (n, f(n))$.
\begin{enumerate}[(a)]
\item Explique por qué cualquier $k$ porciones permiten reconstruir $f(x)$ mediante interpolación de Lagrange en $\mathbb{Z}_p$, y por tanto recuperar $s = f(0)$.
\item Explique por qué con solo $k-1$ porciones, el secreto $s$ puede ser cualquier valor en $\mathbb{Z}_p$ con igual probabilidad.
\textbf{Sugerencia}: Muestre que existe un único polinomio de grado $\le k-1$ compatible con la información parcial y un candidato a secreto $\tilde{s} \in \mathbb{Z}_p$ dado. Concluya el resultado a partir del hecho de que los coeficientes se eligieron uniformemente al azar.
\end{enumerate}
\end{exercise}

\begin{exercise}[Códigos de Reed-Solomon]
Un código de Reed-Solomon $\text{RS}(n,k)$ sobre $\mathbb{Z}_p$ codifica un mensaje $(m_0, \ldots, m_{k-1}) \in \mathbb{Z}_p^k$ como las evaluaciones del polinomio $m(x) = m_0 + m_1 x + \cdots + m_{k-1}x^{k-1}$ en $n$ puntos prefijados $\alpha_1, \ldots, \alpha_n \in \mathbb{Z}_p$, produciendo la palabra código $c = (m(\alpha_1), \ldots, m(\alpha_n))$. 

Suponga que se recibe un vector $r \in \mathbb{Z}_p^n$ que difiere de la verdadera palabra código $c \in \mathbb{Z}_p^n$ en a lo sumo $t$ posiciones (desconocidas). Se definen dos polinomios desconocidos:
\begin{itemize}
\item El polinomio localizador de errores $E(x) = e_0 + e_1 x + \cdots + e_{t-1}x^{t-1} + x^t$ (mónico de grado $t$), cuyas raíces son exactamente los puntos de evaluación $\alpha_j$ donde ocurrieron errores.
\item El polinomio $N(x) = m(x)\cdot E(x) = n_0 + n_1 x + \cdots + n_{k-1+t}\,x^{k-1+t}$, de grado $\leq k - 1 + t$.
\end{itemize}

\begin{enumerate}[(a)]
\item Demuestre que para todo $i = 1, \ldots, n$ vale la ``ecuación clave'' de la corrección de errores:
\[
r_i \cdot E(\alpha_i) = N(\alpha_i).
\]
\textbf{Sugerencia}: Distinga dos casos. Si no hay error en la posición $i$, entonces $r_i = m(\alpha_i)$. Si hay error en la posición $i$, entonces $\alpha_i$ es raíz de $E$.

\item Demuestre que si $n \geq k + 2t$, el mensaje $m(x)$ queda unívocamente determinado por el vector recibido $r$ y la ecuación clave. Concluya que se pueden corregir hasta $t \leq \lfloor(n-k)/2\rfloor$ errores.

\textbf{Sugerencia}: Suponga que $(E, N)$ y $(E', N')$ son dos soluciones de la ecuación clave y considere $P(x) = N(x)E'(x) - N'(x)E(x)$. Demuestre que $P(\alpha_i) = 0$ para todo $i$, acote $\deg(P)$, y use el Ejercicio 15(b) para concluir que $P \equiv 0$. ¿Qué implica esto sobre el cociente $N/E$?

\item Observe que la ecuación clave es lineal en los coeficientes (desconocidos) de $E$ y $N$. Escriba explícitamente el sistema $A\,\mathbf{x} = \mathbf{b}$ en $\mathbb{Z}_p$ de tamaño $n \times (k+2t)$ que permite encontrar los coeficientes de $E$ y $N$ a partir de $r$ y los puntos de evaluación $\alpha_i$. 

\item Usando el resultado de (b), deduzca que el sistema lineal tiene solución única. ¿Cuál es la complejidad de resolver el sistema? ¿Depende de $p$? ¿Tiene sentido hablar del número de condición en aritmética modular?
\end{enumerate}
\end{exercise}



\end{document}






