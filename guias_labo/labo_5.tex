\documentclass[12pt]{article}
\usepackage{graphicx,amsmath,amsfonts,amssymb,epsfig,euscript,enumerate}
\usepackage[T1]{fontenc}
\usepackage[utf8x]{inputenc}

\newtheorem{exercise}{Ejercicio}
\newcommand{\bej}{\begin{exercise}\rm}
\newcommand{\fej}{\end{exercise}}

\newcommand{\R}{\mathbb{R}}
\newcommand{\C}{\mathbb{C}}
\def\dt{\Delta t}
\def\dx{\Delta x}

\topmargin-2cm \vsize 29.5cm \hsize 21cm
\setlength{\textwidth}{16.75cm}\setlength{\textheight}{23.5cm}
\setlength{\oddsidemargin}{0.0cm}
\setlength{\evensidemargin}{0.0cm}

\begin{document}
\centerline{{\small Universidad de Buenos Aires - Facultad de Ciencias Exactas y Naturales - Depto. de Matemática}}
 
 \vskip 0.2cm
 \hrulefill
 \vskip 0.2cm

 \centerline{{\bf\Huge {\sc Elementos de Cálculo Numérico}}}
 \vskip 0.2cm
 \centerline{\ttfamily Primer Cuatrimestre 2026}
 \hrulefill

 \bigskip
 \centerline{\bf Laboratorio N$^\circ$ 6: Ecuaciones Diferenciales Ordinarias}
 \bigskip

\begin{exercise}[Capa límite]\label{ejer_boundary_layer}
Considere la ecuación:
\begin{equation}
\label{boundary_layer_ej}
\varepsilon u_{xx} - u_{x} = f, \hspace{1cm} u(0)=\alpha, \quad u(1)=\beta,
\end{equation}
cuya solución exacta para el caso $f(x) = -1$ está dada por:
\[
u_{\varepsilon}(x) = \alpha + x + (\beta - \alpha -1) \left( \frac{e^{x/\varepsilon}-1 }{e^{1/\varepsilon}-1} \right).
\]
\begin{enumerate}
\item Grafique la solución exacta para el caso $\alpha=1$, $\beta=3$ a medida que $\varepsilon \to 0$. Interprete el significado del término \textit{capa límite} que se suele aplicar al comportamiento de $u_\varepsilon(x)$ para $x$ cerca del borde $\{ x=1 \}$ y $\varepsilon \to 0$. ¿De qué tamaño es la \textit{capa límite}?
\item Resuelva numéricamente la ecuación \eqref{boundary_layer_ej} usando diferencias centradas para las derivadas primera y segunda, y una malla de tamaño $h$. Grafique el error para distintos valores de $h$ y $\varepsilon$. ¿Qué ocurre si $h \gg 2\varepsilon$?
\end{enumerate}
\end{exercise}
 

\end{document}