\documentclass[12pt]{article}
\usepackage{graphicx,amsmath,amsfonts,amssymb,epsfig,euscript,enumerate}
\usepackage[T1]{fontenc}
\usepackage[utf8x]{inputenc}

\newtheorem{exercise}{Ejercicio}
\newcommand{\bej}{\begin{exercise}\rm}
\newcommand{\fej}{\end{exercise}}

\newcommand{\R}{\mathbb{R}}
\newcommand{\C}{\mathbb{C}}
\def\dt{\Delta t}
\def\dx{\Delta x}

\topmargin-2cm \vsize 29.5cm \hsize 21cm
\setlength{\textwidth}{16.75cm}\setlength{\textheight}{23.5cm}
\setlength{\oddsidemargin}{0.0cm}
\setlength{\evensidemargin}{0.0cm}

\begin{document}
\centerline{{\small Universidad de Buenos Aires - Facultad de Ciencias Exactas y Naturales - Depto. de Matemática}}
 
 \vskip 0.2cm
 \hrulefill
 \vskip 0.2cm

 \centerline{{\bf\Huge {\sc Elementos de Cálculo Numérico}}}
 \vskip 0.2cm
 \centerline{\ttfamily Primer Cuatrimestre 2026}
 \hrulefill

 \bigskip
 \centerline{\bf Laboratorio N$^\circ$ 2: Cuadrados Mínimos}
 \bigskip

\bej
\textbf{(Ajuste polinomial por cuadrados mínimos.)}

Dados $N$ datos $(x_i, y_i)$, queremos encontrar el polinomio $p(x) = c_0 + c_1 x + \cdots + c_n x^n$ de grado $n < N$ que minimiza el error cuadrático:
\[
\min_{c \in \R^{n+1}} \| A c - y \|_2^2,
\]
donde $A \in \R^{N \times (n+1)}$ es la matriz de Vandermonde con $A_{ij} = x_i^j$ e $y = (y_1, \ldots, y_N)^T$.

\begin{enumerate}
\item Genere $N = 50$ datos con $x_i$ equiespaciados en $[0, 1]$ e $y_i = \sin(2\pi x_i) + \varepsilon_i$, donde $\varepsilon_i \sim \mathcal{N}(0, 0.2^2)$ es ruido gaussiano. Grafique los datos.

\item Para un grado $n$ dado, arme la matriz $A$ y resuelva el problema de cuadrados mínimos de tres maneras distintas:
\begin{enumerate}[i.]
\item \textbf{Ecuaciones normales}: resuelva $A^T A\, c = A^T y$ usando \texttt{numpy.linalg.solve}.
\item \textbf{Factorización QR}: calcule $A = QR$ con \texttt{numpy.linalg.qr} y resuelva $Rc = Q^T y$.
\item \textbf{SVD}: use \texttt{numpy.linalg.lstsq} (que internamente usa la SVD).
\end{enumerate}
Verifique que los tres métodos dan (aproximadamente) el mismo resultado para $n = 5$.

\item Repita para $n = 5, 10, 15, 20$. Grafique el polinomio ajustado junto con los datos en cada caso. ¿A partir de qué grado se observa sobreajuste (overfitting)?

\item Para cada $n$, calcule el número de condición $\kappa_2(A)$ y el número de condición $\kappa_2(A^T A)$. ¿Qué relación hay entre ambos? ¿Para qué valores de $n$ las ecuaciones normales dejan de dar resultados confiables? Compare con los resultados de QR y SVD.

\item Grafique el error $\|c_{\text{normales}} - c_{\text{SVD}}\|_\infty$ en función de $n$. ¿A partir de qué grado los métodos dan resultados significativamente distintos?
\end{enumerate}
\fej

\end{document}
