\documentclass[12pt]{article}
\usepackage{graphicx,amsmath,amsfonts,amssymb,epsfig,euscript,enumerate}
\usepackage[T1]{fontenc}
\usepackage[utf8x]{inputenc}

\newtheorem{exercise}{Ejercicio}
\newcommand{\bej}{\begin{exercise}\rm}
\newcommand{\fej}{\end{exercise}}

\newcommand{\R}{\mathbb{R}}
\newcommand{\C}{\mathbb{C}}
\def\dt{\Delta t}
\def\dx{\Delta x}

\topmargin-2cm \vsize 29.5cm \hsize 21cm
\setlength{\textwidth}{16.75cm}\setlength{\textheight}{23.5cm}
\setlength{\oddsidemargin}{0.0cm}
\setlength{\evensidemargin}{0.0cm}

\begin{document}
\centerline{{\small Universidad de Buenos Aires - Facultad de Ciencias Exactas y Naturales - Depto. de Matemática}}
 
 \vskip 0.2cm
 \hrulefill
 \vskip 0.2cm

 \centerline{{\bf\Huge {\sc Elementos de Cálculo Numérico}}}
 \vskip 0.2cm
 \centerline{\ttfamily Primer Cuatrimestre 2026}
 \hrulefill

 \bigskip
 \centerline{\bf Laboratorio N$^\circ$ 4: Teoría de Aproximación}
 \bigskip

\bej
Considere la ecuación diferencial $ -u'' + u = f(x) $ con condiciones de borde periódicas: $u(x)=u(x+2\pi)$. 
\begin{itemize}
\item Escriba un código que, dado un lado derecho $f$, calcule su serie de Fourier aproximando la integral mediante la regla de trapecios compuesta.
\item Utilizando que el operador $-u'' + u$ se puede representar en la base de Fourier como una matriz diagonal, despeje la expansión correspondiente de la solución $u=u^N$ en dicha base. 
\item Pruebe su código para los casos $f=f_1=10\sin^3|x-\pi| - 6\sin|x-\pi|$, y $f=f_2=e^{\sin(x)}(\sin^2(x) + \sin(x))$.
\item Grafique el máximo error absoluto entre lo obtenido y las soluciónes exactas $u_1=\sin^3|x-\pi|$ y $u_2=e^{\sin(x)}$ en función de $N$ y en escala log-log (el gráfico debería ser una recta), y compare las pendientes obtenidas con el grado de regularidad de las soluciones.
\end{itemize}
\fej

\end{document}