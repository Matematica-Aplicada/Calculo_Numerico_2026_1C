\documentclass[12pt]{article}
\usepackage{graphicx,amsmath,amsfonts,amssymb,epsfig,euscript,enumerate}
\usepackage[T1]{fontenc}
\usepackage[utf8x]{inputenc}

\newtheorem{exercise}{Ejercicio}
\newcommand{\bej}{\begin{exercise}\rm}
\newcommand{\fej}{\end{exercise}}

\newcommand{\R}{\mathbb{R}}
\newcommand{\C}{\mathbb{C}}
\def\dt{\Delta t}
\def\dx{\Delta x}

\topmargin-2cm \vsize 29.5cm \hsize 21cm
\setlength{\textwidth}{16.75cm}\setlength{\textheight}{23.5cm}
\setlength{\oddsidemargin}{0.0cm}
\setlength{\evensidemargin}{0.0cm}

\begin{document}
\centerline{{\small Universidad de Buenos Aires - Facultad de Ciencias Exactas y Naturales - Depto. de Matemática}}
 
 \vskip 0.2cm
 \hrulefill
 \vskip 0.2cm

 \centerline{{\bf\Huge {\sc Elementos de Cálculo Numérico}}}
 \vskip 0.2cm
 \centerline{\ttfamily Primer Cuatrimestre 2026}
 \hrulefill

 \bigskip
 \centerline{\bf Laboratorio N$^\circ$ 1: Introducción a la Programación}
 \bigskip

\bej
\textbf{(Inestabilidad numérica de una recurrencia.)}
Defina $I_n = \int_0^1 x^n e^x\, dx$ para $n \geq 0$.
\begin{enumerate}
\item Calcule $I_0$ exactamente.
\item Integrando por partes, demuestre que $I_n = e - n\, I_{n-1}$ para $n \geq 1$.
\item Programe la recurrencia y calcule $I_n$ para $n = 1, 2, \ldots, 25$. ¿Qué observa? ¿Los valores obtenidos son razonables? (Observe que $0 < I_n < e$ para todo $n$, y que $I_n \to 0$.)
\end{enumerate}
\fej

\bej
\textbf{(Año bisiesto.)}
En el calendario gregoriano, un año es bisiesto si es divisible por 4, \textit{excepto} los años divisibles por 100, que no son bisiestos, \textit{salvo} que también sean divisibles por 400, en cuyo caso sí lo son.
\begin{enumerate}
\item Escriba una función \texttt{es\_bisiesto(n)} que dado un entero $n$ devuelva \texttt{True} si el año $n$ es bisiesto y \texttt{False} en caso contrario.
\item Verifique que su función da los resultados correctos para los años 1900 (no bisiesto), 2000 (bisiesto), 2024 (bisiesto) y 2025 (no bisiesto).
\item Use su función para contar cuántos años bisiestos hay entre 1 y 2026.
\end{enumerate}
\fej

\bej
\textbf{(Conjetura de Collatz.)}
Dado un entero positivo $n$, defina la sucesión:
\[
a_{k+1} = \begin{cases} a_k / 2 & \text{si } a_k \text{ es par,} \\ 3a_k + 1 & \text{si } a_k \text{ es impar,} \end{cases}
\]
con $a_0 = n$. La conjetura de Collatz afirma que esta sucesión siempre llega al valor 1, independientemente del valor inicial $n$.
\begin{enumerate}
\item Escriba una función \texttt{collatz(n)} que devuelva la lista de valores de la sucesión desde $a_0 = n$ hasta el primer $a_k = 1$.
\item Grafique la sucesión para $n = 27$. ¿Cuántos pasos tarda en llegar a 1? ¿Cuál es el valor máximo que alcanza?
\item Verifique la conjetura para todo $n$ entre 1 y 10000. Para cada $n$, registre la cantidad de pasos hasta llegar a 1. Grafique esta cantidad en función de $n$.
\end{enumerate}
\fej

\bej
\textbf{(Búsqueda binaria.)}
Dada una lista ordenada $a_0 \leq a_1 \leq \cdots \leq a_{n-1}$ y un valor $v$, la búsqueda binaria encuentra un índice $i$ tal que $a_i = v$ (o determina que $v$ no está en la lista) comparando $v$ con el elemento central y descartando la mitad de la lista en cada paso.
\begin{enumerate}
\item Escriba una función \texttt{busqueda\_binaria(lista, v)} que devuelva el índice $i$ tal que $\texttt{lista[i]} = v$, o $-1$ si $v$ no está en la lista.
\item Pruebe su función buscando varios valores en la lista $[2, 5, 8, 12, 16, 23, 38, 42, 77, 91]$.
\end{enumerate}
\fej

\end{document}
