\documentclass[12pt]{article}
\usepackage{graphicx,amsmath,amsfonts,amssymb,epsfig,euscript,enumerate}
\usepackage[T1]{fontenc}
\usepackage[utf8x]{inputenc}

\newtheorem{exercise}{Ejercicio}
\newcommand{\bej}{\begin{exercise}\rm}
\newcommand{\fej}{\end{exercise}}

\newcommand{\R}{\mathbb{R}}
\newcommand{\C}{\mathbb{C}}
\def\dt{\Delta t}
\def\dx{\Delta x}

\topmargin-2cm \vsize 29.5cm \hsize 21cm
\setlength{\textwidth}{16.75cm}\setlength{\textheight}{23.5cm}
\setlength{\oddsidemargin}{0.0cm}
\setlength{\evensidemargin}{0.0cm}

\begin{document}
\centerline{{\small Universidad de Buenos Aires - Facultad de Ciencias Exactas y Naturales - Depto. de Matemática}}
 
 \vskip 0.2cm
 \hrulefill
 \vskip 0.2cm

 \centerline{{\bf\Huge {\sc Elementos de Cálculo Numérico}}}
 \vskip 0.2cm
 \centerline{\ttfamily Primer Cuatrimestre 2026}
 \hrulefill

 \bigskip
 \centerline{\bf Laboratorio N$^\circ$ 3: Convolución y FFT}
 \bigskip

\bej
\textbf{(Convolución circular.)}
Dados dos vectores $u, v \in \C^N$, su convolución circular es el vector $w \in \C^N$ con
\[
w_k = \sum_{j=0}^{N-1} u_j\, v_{(k-j) \bmod N}, \quad k = 0, \ldots, N-1.
\]

\begin{enumerate}
\item Programe una función \texttt{conv\_circular\_directa(u, v)} que calcule $w$ usando la definición (dos bucles anidados, o un bucle y slicing). ¿Cuál es la complejidad de este algoritmo?

\item Programe una función \texttt{conv\_circular\_fft(u, v)} que calcule $w$ usando la FFT:
\begin{enumerate}[i.]
\item Calcular $\hat{u} = \texttt{fft}(u)$ y $\hat{v} = \texttt{fft}(v)$.
\item Multiplicar punto a punto: $\hat{w}_k = \hat{u}_k \cdot \hat{v}_k$.
\item Recuperar $w = \texttt{ifft}(\hat{w})$.
\end{enumerate}
¿Cuál es la complejidad de este algoritmo?

\item Genere vectores aleatorios $u, v \in \R^N$ para $N = 2^{10}$ y verifique que ambas funciones dan el mismo resultado (salvo errores de redondeo). ¿Cuál es el máximo de $|w_{\text{directa}} - w_{\text{fft}}|$?

\item Mida el tiempo de ejecución de ambas funciones para $N = 2^8, 2^{10}, 2^{12}, 2^{14}, 2^{16}$. Grafique los tiempos en escala log-log. ¿Se observa la diferencia $O(N^2)$ vs.\ $O(N \log N)$?
\end{enumerate}
\fej

\bej
\textbf{(Convolución lineal via FFT.)}
Dados $u \in \C^M$ y $v \in \C^K$, su convolución lineal es el vector $w \in \C^{M+K-1}$ con
\[
w_k = \sum_{j} u_j\, v_{k-j}, \quad k = 0, \ldots, M+K-2,
\]
donde $u_j = 0$ para $j \notin \{0, \ldots, M-1\}$ y $v_j = 0$ para $j \notin \{0, \ldots, K-1\}$.

\begin{enumerate}
\item Programe una función \texttt{conv\_lineal\_directa(u, v)} que calcule $w$ usando la definición.

\item Para calcular la convolución lineal usando FFT, hay que realizar \textit{zero-padding}: extender ambos vectores a longitud $N \geq M + K - 1$ completando con ceros.
\begin{enumerate}[i.]
\item Explique por qué si $N < M + K - 1$, la convolución circular de los vectores extendidos \textbf{no} coincide con la convolución lineal (se produce \textit{aliasing temporal}).
\item Programe una función \texttt{conv\_lineal\_fft(u, v)} que haga el zero-padding adecuado, aplique la convolución circular via FFT, y devuelva los primeros $M+K-1$ elementos.
\end{enumerate}

\item Genere $u \in \R^{100}$ y $v \in \R^{50}$ aleatorios. Verifique que \texttt{conv\_lineal\_fft} da el mismo resultado que \texttt{conv\_lineal\_directa}. Compare también con \texttt{numpy.convolve(u, v)}.

\item Ilustre el efecto del aliasing: calcule la convolución circular (sin zero-padding) de $u$ y $v$ extendidos a $N = \max(M, K)$ y compare con la convolución lineal correcta. Grafique ambos resultados superpuestos. ¿Dónde se concentran las diferencias?

\item Mida el tiempo de ejecución de ambos métodos para $M = K = 2^8, 2^{10}, 2^{12}, 2^{14}$. ¿A partir de qué tamaño la versión FFT es más rápida?
\end{enumerate}
\fej

\end{document}
